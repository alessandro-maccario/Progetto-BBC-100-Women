\documentclass[11pt, a4paper]{report}

\usepackage[utf8]{inputenc}
\usepackage[T1]{fontenc}
\usepackage{authblk}
\usepackage{sectsty,lmodern} % per modificare il font dei titoli dei capitoli
\usepackage[margin=0.7in]{geometry} % set the margins
\usepackage{graphicx} % for images
\usepackage{pdfpages} % serve per poi inserire la frontpage magari creata con Scribus
\usepackage[english, italian]{babel}
\usepackage{tcolorbox} % per creare il box dell'abstract
\usepackage{capt-of}%%To get the caption
\usepackage{amsmath}
\usepackage{caption}
\usepackage{tabularx}
\usepackage{float}
\usepackage{enumitem}
\usepackage{ctable} % for \specialrule command
\usepackage{blindtext}
\usepackage{hyperref} % per inserire link cliccabili all'interno del testo
\usepackage{url}            % simple URL typesetting



\title{\line(1,0){530}\\ \vspace{10px} Università degli Studi di Milano-Bicocca \\
	PROGETTO DATA MANAGEMENT \& DATA VISUALIZATION \line(1,0){530}\\ \vspace{-5px}}

\author[*]{Giorgia Antonicelli 000000}
\author[*]{Lorenzo Lorgna 000000}
\author[*]{Alessandro Maccario 865682}

\date{} % clear date per non inserire ogni volta la data di running del file
\affil[*]{CdLM Data Science, Università degli Studi di Milano Bicocca}

\renewcommand\Authands{, }

\chapternumberfont{\fontsize{35pt}{32pt}\selectfont}
\chaptertitlefont{\fontsize{12pt}{18pt}\selectfont}


\begin{document}
\maketitle
	
	% Una volta creata la frontpage con Scribus e inserita nella stessa cartella in cui si ha questo stesso file è possibile richiamarlo in "includepdf" tramite il titolo inserendola come prima pagina
	%\begin{titlepage}
	%	\includepdf{title}
	%\end{titlepage}
	
\tableofcontents
	
\selectlanguage{english} % Necessario per avere il titolo "Abstract" e non "Sommario" in italiano 
	
	
%------------------------------------------------------------------------------------------------
\newpage
%------------------------------------------------------------------------------------------------

	
\begin{tcolorbox}[colframe=blue!75!black]
	\begin{abstract}
		QUI CI VA L'ABSTRACT!!!
	\end{abstract}
\end{tcolorbox}
	
	\selectlanguage{italian} % necessario per avere il titolo "Riferimenti bibliografici" in italiano e non come "References"
	
	\vspace{15px}
	
	
	\chapter{Data Management}
	
	\chapter[Data Visualization]{Progettazione, implementazione e valutazione delle visualizzazioni}
	
	%--------------  PROGETTAZIONE --------------%
	\section{Progettazione}
	
	L'intenzione alla base del progetto è stata quella di valutare l'impatto che la lista della BBC ha sulla popolarità, tramite Twitter, delle donne nominate. I dati scaricati dal social network sono poi stati integrati grazie ad altre fonti per ottenere diverse \textit{feature} in grado di fornire un quadro più ampio del fenomeno in studio, come ad esempio quello di analizzare l'evento nel corso degli anni dal 2015 al 2019. Ci si è poi concentrati su due anni in particolare, ovvero il primo e l'ultimo. Tale scelta è stata basata sul considerare sufficiente un periodo temporale distanziati di quattro anni (e non cinque a causa del particolare anno del COVID) e in grado di fornire un'immagine dei cambiamenti in termini di interesse della comunità femminile presente su Twitter rispetto all'evento. \\
	
	Concluso il lavoro di raccolta, pulizia, elaborazione e analisi dei dati, le infografiche che vengono di seguito presentate ne sono il risultato ottenute tramite l'utilizzo del software \textit{Tableau}. Una volta valutate tramite una \textit{valutazione euristica}, un \textit{questionario psicometrico} (nello specifico, è stato utilizzato il Cabitza-Locoro) e uno \textit{user test} sono stati ottenuti i grafici di valutazione della qualità riportati di seguito e creati utilizzando diversi script in linguaggio \textit{R}. \\
	
	Complessivamente, si sono visualizzati $ 5 $ aspetti di interesse:
	
	\begin{itemize}
		\item Infografica 1: Heatmap, per mostrare l'andamento del numero di tweet nel corso degli anni dal 2015 al 2019;
		\item Infografica 2: Lollipop, per analizzare da un punto di vista descrittivo la composizione delle categorie stabilite dalla BBC, l'età mediana e il numero di donne per ciascuna categoria;
		\item Infografica 3: Alluvial diagram, per visualizzare la relazione tra categorie, provenienza geografica e popolarità delle donne;
		\item Infografica 4: Scatterplot, per mostrare più nello specifico gli stati di appartenenza delle donne nominate nella lista e i dati relativi ai paesi di provenienza. Vengono nominate più donne provenienti da stati ricchi/più sviluppati o vicevera?
		\item Infografica 5: Boxplot/Barplot, per indicare l'impatto della nomination rispetto al numero di tweet per ciascuna donna pre-pubblicazione e post-pubblicazione.
	\end{itemize}
	
	In sintesi, le domande di ricerca a cui si è voluto rispondere anche tramite le visualizzazioni sono state:
	
	\begin{enumerate}
		\item 
		\item 
		\item 
	\end{enumerate}
	
	In una prima fase di progettazione, non molto dissimile dal risultato finale, venivano visualizzati il numero di tweet per singolo giorno nell'anno $ 2015 $ o $ 2019 $. Il colore rosa aveva l'obiettivo di indicare il mondo femminile e richiamava, almeno in parte, il colore utilizzato come tema per l' \href{https://www.internationalwomensday.com/}{International Women's Day}. La visualizzazione era stata corredata del context necessario a rendere chiari alcuni istanti temporali ben precisi a sottolineare la colorazione più intensa dovuta all'aumento del numero di tweets della giornata, tutti incentrati su uno specifico tema del mondo femminile.
	
	\begin{figure}[h!]
		\centering
		\includegraphics[totalheight=8cm]{heatmap_2019.png}
		\caption{Prima fase di progettazione (esempio della Heatmap)}
		\label{fig:prima_fase_progettazione_esempio_heatmap}
	\end{figure}
	
	Lo stesso procedimento è stato applicato anche all'Alluvial diagram INSERIRE ALL'ALLUVIAL DIAGRAM DELLA PRIMA ORA E INSERIRCI UN COMMENTO, COME ANCHE PER ESEMPIO LO SCATTERPLOT DELLA PRIMA ORA CON UN COMMENTO. DOPODICHé PASSARE DIRETTAMENTE ALLA FASE DI IM
	
		
	%--------------  IMPLEMENTAZIONE --------------%
	\section{Implementazione}
	


	%--------------  VALUTAZIONE --------------%
	\section{Valutazione}
	
	QUI INSERI


	
	%--------------  VALUTAZIONE EURISTICA --------------%
	
	\subsection{Valutazione Euristica}
	
	
	%--------------  QUESTIONARIO PSICOMETRICO --------------%
	
	\subsection{Questionario Psicometrico}
	
	%--------------  USER TEST --------------%
	
	\subsection{User test}
	
	
	
	
	
	
	
	
	
	
	
	
	
	
	
	
	
	
	
	
	
	
	
	
%------------------------------------------------------------------------------------------------
\newpage
%------------------------------------------------------------------------------------------------


\begin{thebibliography}{100}
	
\bibitem{Country codes}
Country Code
\url{https://www.iban.com/country-codes}

\bibitem{BBC Women 2020}
BBC 100 Women
\url{https://raw.githubusercontent.com/rfordatascience/tidytuesday/master/data/2020/2020-12-08/women.csv}

\bibitem{GDP Country}
GDP Country
\url{https://www.imf.org/external/datamapper/NGDPD@WEO/OEMDC/ADVEC/WEOWORLD}

\bibitem{GDP per capita Country}
GDP per capita Country
\url{https://www.imf.org/external/datamapper/NGDPDPC@WEO/OEMDC/ADVEC/WEOWORLD}

\bibitem{Labour percentage}
GDP per capita Country
\url{https://databank.worldbank.org/source/gender-statistics/preview/on}

\bibitem{Gender Global Gap 2015}
Gender Global Gap 2020
\url{https://reports.weforum.org/global-gender-gap-report-2015/rankings/}

\bibitem{Gender Global Gap 2020}
Gender Global Gap 2020
\url{https://reports.weforum.org/global-gender-gap-report-2020/the-global-gender-gap-index-2020-rankings/}

\bibitem{HDI}
Human Development Index
\url{http://hdr.undp.org/en/indicators/137506}	
	
	
	
	
\end{thebibliography}


	
	
\end{document}